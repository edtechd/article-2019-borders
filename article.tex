\thispagestyle{empty}
{\theme{Analysis of border detection methods effectiveness for accuracy of CT image segmentation.}}

\vspace{5mm}

\begin{center} Kirill K. Kazakhmedov $^{1}$, Eduard B. Demishkevich$^{1,2}$, Sergey S. Gavriushin$^{1,2}$  \\

\vspace{5mm}

{\small $^{1}$ Bauman Moscow State Technical University, 5c1, 2nd Baumanskaya st., 105005, Moscow, Russian Federation \\
$^{2}$ Mechanical Engineering Research Institute of the Russian Academy of Sciences, 4, Malyi Kharitonievsky pereulok, 101990, Moscow, Russian Federation \\
mail@edtech.ru }
\end{center}

\begin{changemargin}{0.5cm}{0.5cm}
{\small \textbf{Abstract.} The objective of this study was to compare different techniques of border detection in layerwise images obtained with use of 
computed tomography. Border detection is used to pick certain objects on images then reconstruct 3d-models from CT slices for further biomechanical analysis.
The surveyed techniques are the Sobel operator, the Prewitt operator, the Roberts cross operator, the Canny operator and the adaptive binarization. 
The study contains a review of metrics to evaluate segmentation quality.}

{\small \textbf{Keywords:} image processing, computed tomography}

\end{changemargin}

\ctheme{1. Introduction}

% описать современный подход к индивидуализированному лечению: по томограммам строится МКЭ модель, затем производится биомеханический расчёт,

% описать основные этапы построения 3D модели по томограмме

% указать проблемы при сегментации

% сослаться на предыдущие работы и указать, что целью работы является сравнение методик улучшения изображений, чтобы сделать вывод и необходимости 
% использования того или иного способа

\ctheme{2. Materials and methods}

% Описать суть исследуемых алгоритмов

% Описать основные метрики оценки качества сегментации

\ctheme{3. Results}

% Описать как проводилось исследование алгоритмов и какие получены результаты

\ctheme{4. Discussion}

% Какие могут быть сделаны выводы из результатов?

\ctheme{5. Conclusion}

% вкратце описать, что было сделано

% указать, что проведенное исследование позволяет сделать вывод о необходимости использования той или иной техники улучшения изображений

% что может быть сделано в дальнейших исследованиях?

\ctheme{References}

A Comparison of Filtering Techniques for Image Quality
Improvement in Computed Tomography 
https://pdfs.semanticscholar.org/40ac/47fce898f5b7f581f8fce41e882755c26c0b.pdf

